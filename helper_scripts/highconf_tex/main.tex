\documentclass[10pt,a4paper,twocolumn]{article}

\usepackage[table]{xcolor}
\usepackage{caption}
\usepackage{subcaption}
\usepackage{graphicx}

\begin{document}

% rq1
\newcommand{\domsubgraph}[2]{
\begin{subfigure}{.55\linewidth}
    \centering
    \includegraphics[width=\linewidth]{rq1/#1_counts_high_conf.png}
    \caption{#2}
  \end{subfigure}
}

% rq3 mannwhitney
\newcommand{\mwgraphsimple}[2]{
\begin{figure*}
    \centering
    \makebox[\textwidth][c]{
        \begin{subfigure}{\linewidth}
            \centering
            \includegraphics[width=\linewidth]{rq3_mw/#1_simple_high_conf_plot_arrows.png}
            \caption{Grouped by decision type first and subgrouped by domain.}
        \end{subfigure}
    }

    \makebox[\textwidth][c]{
        \begin{subfigure}{\linewidth}
            \centering
            \includegraphics[width=\linewidth]{rq3_mw/#1_simple_inverted_high_conf_plot_arrows.png}
            \caption{Grouped by domain first and subgrouped by decision type. (CM = Content Management, DSP = Data Storage and Processing, DC = DevOps and Cloud, SOAM = SOA and Middlewares, SDT = Software Development Tools, WD = Web Development)}
        \end{subfigure}
    }

    \caption{Mann-Whitney test on the relations between #2, domain, and decision type contained in the issue, simple decision view.}
    \label{fig:rq3_#1}
\end{figure*}
}

\newcommand{\mwgraphintersected}[2]{
\begin{figure*}
    \centering
    \makebox[\textwidth][c]{
        \begin{subfigure}{\linewidth}
            \centering
            \includegraphics[width=\linewidth]{rq3_mw/#1_high_conf_plot_arrows.png}
            \caption{Grouped by decision type first and subgrouped by domain.}
        \end{subfigure}
    }

    \makebox[\textwidth][c]{
        \begin{subfigure}{\linewidth}
            \centering
            \includegraphics[width=\linewidth]{rq3_mw/#1_inverted_high_conf_plot_arrows.png}
            \caption{Grouped by domain first and subgrouped by decision type. (CM = Content Management, DSP = Data Storage and Processing, DC = DevOps and Cloud, SOAM = SOA and Middlewares, SDT = Software Development Tools, WD = Web Development)}
        \end{subfigure}
    }

    \caption{Mann-Whitney test on the relations between #2, domain, and decision type contained in the issue, intersected decision view.}
    \label{fig:rq3_#1}
\end{figure*}
}

% rq2 chisq
\newcommand{\domaintable}[3]{
\begin{#3}[]
    \centering
\input{rq2_chisq/table_#1_high_conf.tex}
    \caption{Chi-squared test results on the #2 characteristic per domain. (RQ2) (CM = Content Management, DSP = Data Storage and Processing, DC = DevOps and Cloud, SOAM = SOA and Middlewares, SDT = Software Development Tools, WD = Web Development)}
    \label{tab:rq2_#1}
\end{#3}
}

\newcommand{\chithreeone}[2]{
\begin{table*}[]
    \centering
    \small
\input{rq3_chisq/#1_intersected_high_conf.tex}
    \caption{Chi-squared test results for the relation between the #2 issue characteristic and simple decision type contained in the issue, RQ3. \textcolor{blue}{intersected} (CM = Content Management, DSP = Data Storage and Processing, DC = DevOps and Cloud, SOAM = SOA and Middlewares, SDT = Software Development Tools, WD = Web Development)}
    \label{tab:rq3_#1}
\end{table*}
}
\newcommand{\chithreetwo}[2]{
\begin{table}[]
    \centering
    \small
\input{rq3_chisq/#1_simple_high_conf.tex}
    \caption{Chi-squared test results for the relation between the #2 issue characteristic and simple decision type contained in the issue, RQ3. \textcolor{blue}{simple} (CM = Content Management, DSP = Data Storage and Processing, DC = DevOps and Cloud, SOAM = SOA and Middlewares, SDT = Software Development Tools, WD = Web Development)}
    \label{tab:rq3_#1}
\end{table}
}
\newcommand{\chithreethree}[2]{
    \begin{table}[]
        \centering
        \small
    \input{rq3_chisq/#1_simple_high_conf_full.tex}
        \caption{Chi-squared test results for the relation between the #2 issue characteristic and simple decision type contained in the issue, RQ3. \textcolor{blue}{simple full-domain only} (CM = Content Management, DSP = Data Storage and Processing, DC = DevOps and Cloud, SOAM = SOA and Middlewares, SDT = Software Development Tools, WD = Web Development)}
        \label{tab:rq3_#1}
    \end{table}
}
\newcommand{\chithreefour}[2]{
    \begin{table}[]
        \centering
        \small
    \input{rq3_chisq/#1_simple_high_conf_not_full.tex}
        \caption{Chi-squared test results for the relation between the #2 issue characteristic and simple decision type contained in the issue, RQ3. \textcolor{blue}{simple without full} (CM = Content Management, DSP = Data Storage and Processing, DC = DevOps and Cloud, SOAM = SOA and Middlewares, SDT = Software Development Tools, WD = Web Development)}
        \label{tab:rq3_#1}
    \end{table}
}

% RQ1
\begin{figure}[h]
    \centering
    \includegraphics[width=\linewidth]{rq1/total_counts_high_conf.png}
    \caption{A Venn diagram of the amount of issues divided over the design decision types, across all domains. (RQ1)}
    \label{fig:rq1_totals}
  \end{figure}

\begin{figure*}[!th]
    \centering
    \makebox[\textwidth][c]{
        \domsubgraph{content management}{Content Management}
        \domsubgraph{data storage & processing}{Data Storage \& Processing}
      }
    
    \makebox[\textwidth][c]{
        \domsubgraph{devops and cloud}{DevOps and Cloud}
        \domsubgraph{soa and middlewares}{SOA and Middlewares}
    }
    
    \makebox[\textwidth][c]{
        \domsubgraph{software development tools}{Software Development Tools}
        \domsubgraph{web development}{Web Development}
    }
      
    \caption{Venn diagrams of the amount of issues divided over the design decision types, per domain (RQ1).}
    \label{fig:rq1_domainspecific}
\end{figure*}

\begin{table*}[]
    \centering
\begin{tabular}{|c||c|}
\hline
Domain & Non-Arch \\ 
\hline
CM & \cellcolor[rgb]{0.8458168047029009,0.5361995422603977,0.36009568438937417} 0.69 \\ 
\hline
DSP & \cellcolor[rgb]{0.757516360519213,0.767770907614364,0.41999999999999993} 1.49 \\ 
\hline
DC & \cellcolor[rgb]{0.8996799856839783,0.7911519322374969,0.41036798663837964} 0.95 \\ 
\hline
SOAM & \cellcolor[rgb]{0.8843428615374621,0.8278466186230083,0.42} 1.08 \\ 
\hline
SDT & \cellcolor[rgb]{0.7999577423630035,0.3191333138515498,0.3172938928721366} 0.46 \\ 
\hline
WD & \cellcolor[rgb]{0.8584329567655611,0.5959159953569889,0.371870759647857} 0.75 \\ 
\hline
\end{tabular}
    \caption{Chi-squared test results on decision types per domain, with intersected decision types. (RQ1) (CM = Content Management, DSP = Data Storage and Processing, DC = DevOps and Cloud, SOAM = SOA and Middlewares, SDT = Software Development Tools, WD = Web Development)}
    \label{table:rq1_intersected}
\end{table*}

\begin{table}[]
    \centering
\begin{tabular}{|c||c|c|c|c|}
\hline
Domain & Exis & Exec & Prop & Non-Arch \\ 
\hline
CM & \cellcolor[rgb]{0.8240461478944738,0.4331517667005097,0.33977640470150894} 0.66 & \cellcolor[rgb]{0.7712261139529041,0.18313693937707926,0.2904777063560438} 0.44 & \cellcolor[rgb]{0.8320694322019001,0.4711286457556604,0.3472648033884401} 0.69 & \cellcolor[rgb]{0.8986941818233326,0.8346446124426312,0.42} 1.03 \\ 
\hline
DSP & \cellcolor[rgb]{0.6655400993622476,0.7242032049610647,0.42000000000000004} 1.57 & \cellcolor[rgb]{0.8586739427877126,0.5970566625285062,0.37209567993519843} 0.79 & \cellcolor[rgb]{0.5716911892064963,0.6797484580451826,0.42} 1.79 &  \\ 
\hline
DC & \cellcolor[rgb]{0.8873072469278359,0.732587635458423,0.39882009713264677} 0.91 & \cellcolor[rgb]{0.53,0.66,0.42} 1.89 & \cellcolor[rgb]{0.8403580044364152,0.5103612209990319,0.35500080414065416} 0.72 & \cellcolor[rgb]{0.9015148543368908,0.7998369771946165,0.4120805307144314} 0.97 \\ 
\hline
SOAM &  & \cellcolor[rgb]{0.865749204619536,0.8190390969250433,0.42} 1.10 & \cellcolor[rgb]{0.8771660124927823,0.6845857924658358,0.3893549449932634} 0.87 & \cellcolor[rgb]{0.9090428213237686,0.835469354265838,0.41910663323551733} 1.00 \\ 
\hline
SDT & \cellcolor[rgb]{0.7702281825359908,0.17841339733702288,0.2895463037002581} 0.44 & \cellcolor[rgb]{0.8563315482728722,0.5859693284915951,0.3699094450546807} 0.79 & \cellcolor[rgb]{0.76,0.13,0.28} 0.40 & \cellcolor[rgb]{0.9033084551096389,0.8368303208414078,0.42} 1.02 \\ 
\hline
WD & \cellcolor[rgb]{0.8355624898676486,0.48766245204020336,0.3505249905431387} 0.70 & \cellcolor[rgb]{0.7690458845673809,0.7732322611108646,0.42} 1.33 & \cellcolor[rgb]{0.7924164728825419,0.2834379716440316,0.31025537469037245} 0.53 & \cellcolor[rgb]{0.9077791723204397,0.829488082316748,0.41792722749907707} 0.99 \\ 
\hline
\end{tabular}
    \caption{Chi-squared test results on decision types per domain, with simplified decision types. (RQ1) (CM = Content Management, DSP = Data Storage and Processing, DC = DevOps and Cloud, SOAM = SOA and Middlewares, SDT = Software Development Tools, WD = Web Development)}
    \label{table:rq1_simplified}
\end{table}

% RQ2

\domaintable{hierarchy}{hierarchy}{table}
\domaintable{issue_type}{issue\_type}{table}
\domaintable{resolution}{resolution}{table*}
\domaintable{status}{status}{table*}

\begin{figure*}
    \centering
    \begin{subfigure}{.35\textwidth}
      \centering
      \includegraphics[width=\linewidth]{rq2_mw/domlegend.png}
      \caption{Legend for the rest of the graphs in this figure.}
    \end{subfigure}
    \begin{subfigure}{.4\textwidth}
      \centering
      \includegraphics[width=\linewidth]{rq2_mw/comment avg size_high_conf_plot_arrows.png}
      \caption{Comment Average Size.}
    \end{subfigure}
    
    \begin{subfigure}{.4\textwidth}
      \centering
      \includegraphics[width=\linewidth]{rq2_mw/comment count_high_conf_plot_arrows.png}
      \caption{Comment Count.}
    \end{subfigure}
    \begin{subfigure}{.4\textwidth}
      \centering
      \includegraphics[width=\linewidth]{rq2_mw/description size_high_conf_plot_arrows.png}
      \caption{Description Size.}
    \end{subfigure}
    
    \begin{subfigure}{.4\textwidth}
      \centering
      \includegraphics[width=\linewidth]{rq2_mw/duration_high_conf_plot_arrows.png}
      \caption{Duration.}
    \end{subfigure}
    \begin{subfigure}{.4\textwidth}
      \centering
      \includegraphics[width=\linewidth]{rq2_mw/votes_high_conf_plot_arrows.png}
      \caption{Votes.}
    \end{subfigure}

    \begin{subfigure}{.4\textwidth}
      \centering
      \includegraphics[width=\linewidth]{rq2_mw/watches_high_conf_plot_arrows.png}
      \caption{Watches.}
    \end{subfigure}
      
    \caption{Box plots of the continuous data issue characteristics. Arrows indicate significantly larger means according to the Mann-Whitney test. (RQ2)}
    \label{fig:rq2_contvars}
\end{figure*}

% RQ3

\chithreeone{hierarchy}{hierarchy}
\chithreetwo{hierarchy}{hierarchy}
\chithreethree{hierarchy}{hierarchy}
\chithreefour{hierarchy}{hierarchy}

\chithreeone{issue_type}{issue\_type}
\chithreetwo{issue_type}{issue\_type}
\chithreethree{issue_type}{issue\_type}
\chithreefour{issue_type}{issue\_type}

\chithreeone{resolution}{resolution}
\chithreetwo{resolution}{resolution}
\chithreethree{resolution}{resolution}
\chithreefour{resolution}{resolution}

\chithreeone{status}{status}
\chithreetwo{status}{status}
\chithreethree{status}{status}
\chithreefour{status}{status}

\mwgraphsimple{description size}{description size}
\mwgraphintersected{description size}{description size}

\mwgraphsimple{comment count}{comment count}
\mwgraphintersected{comment count}{comment count}

\mwgraphsimple{comment avg size}{average comment size}
\mwgraphintersected{comment avg size}{average comment size}

\mwgraphsimple{duration}{duration}
\mwgraphintersected{duration}{duration}

\mwgraphsimple{votes}{votes}
\mwgraphintersected{votes}{votes}

\mwgraphsimple{watches}{watches}
\mwgraphintersected{watches}{watches}


\end{document}